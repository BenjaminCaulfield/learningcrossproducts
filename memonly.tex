We have seen that learning with membership queries can be made significantly easier if a single positive example is given. 
In this section we describe a learning algorithm using membership queries when no positive example is given. 
This algorithm makes $O(max_i \{ \memCi{i}(c^*_i) \}^k)$ queries, matching the lower bound given in a previous section. 

For this algorithm to work, we need to assume that $\emptyset \not\in C_i$ for all $i$.
If not, there is no way to distinguish between an empty and non-empty concept. 
For example consider the classes $C_1 = \{ \{1\}, \emptyset \}$ and $C_2 = \{ \{j \} \mid j \in \mathbb{N} \}$. 
It is easy to know when we have learned the correct class in $C_1$ or in $C_2$ using membership queries. 
However, learning from their cross-product is impossible. 
For any finite number of membership queries, there is no way to distinguish between the sets $\emptyset$ and $\{(1,j)\}$ for some $j$ that has yet to be queried.


The main idea behind this algorithm is that learning from membership queries is easy once a single positive example is found. 
So the algorithm runs until a positive example is found from each concept or until all concepts are learned. 
If a positive example is found, the learner can then run the simple algorithm from Proposition \ref{memandpos} for learning from membership queries and a single positive example. 


\begin{proposition}
Algorithm \ref{memonlyalg} will terminate after making $O(max_i \{ \memCi{i}(c^*_i) \}^k)$ queries.
\end{proposition}
\begin{proof}
The algorithm works by constructing sets $T_i$ of elements and querying all possible elements of $\prod T_i$. 
We will get our bound of $O(max_i \{ \memCi{i}(c^*_i) \}^k)$ by showing the algorithm will find a positive example once $| T_i | > max_i \{ \memCi{i}(c^*_i) \}$ for all $i$. 
Since the algorithm queries all possible elements of $\prod T_i$, it is sufficient to prove that $T_i$ will contain an element of $c^*_i$ once $|T_i| > \memCi{i}(c^*_i)$.
We will now show this is true for each $i$.

Assume that the sublearner $A_i$ eventually terminates with the correct answer $c^*_i$. 
%This means that for any $c_i$ in any $C_i$, either $A_i$ eventually queries a point $x \in c_i$ or the learner outputs $c_i$ after finitely many negative membership queries. 
Let $\vec{q_i} := (q_1^i, q_2^i, \dots) \in X_i^*$ be the elements whose membership $A_i$ would query assuming it only received negative answers from an oracle. 
If $\vec{q_i}$ is finite, then there is some set $N_i \in C_i$ that $A_i$ outputs after querying all elements in $\vec{q_i}$ (and receiving negative answers). 
We will consider the cases when $c^*_i = N_i$ and $c^*_i \ne N_i$

Assume $c^*_i = N_i$: Then by our assumption that $\targ \ne \emptyset$, $N_i$ contains some element $n_i$. 
Note that although sampling elements from a set might be expensive in general, this is only done for $N_i$ and can therefore be hard-coded into the learning algorithm. 
The algorithm will start with $T_i := \{n_i\}$, so $T_i$ contains an element of $c^*_i$ at the start of the algorithm. 

Assume $c^*_i \ne N_i$: By our assumption that $A_i$ eventually terminates, $A_i$ must eventually query some $q_j^i \in c^*_i$ (Otherwise, $A_i$ would only receive negative answers and would output $N_i$). 
So after $j$ steps, $T_i$ contains some element of $c^*_i$.
Since $j < \memCi{i}(c^*_i)$, we have that $T_i$ contains a positive example once $| T_i | > \memCi{i}(c^*_i)$, completing the proof.  
\end{proof}






\begin{algorithm}[H]
\label{memonlyalg}
\SetAlgoLined
%\KwResult{Learning with Membership Queries Only}
\For{$i = 1 \dots k$}{
	\eIf{$N_i$ and $n_i$ exist}{
		Set $T_i := \{n_i\}$\;
	}{
		Set $T_i := \{ \}$\; 
	}
}
Set $j := 0$\;
\While{Some $A_i$ has not terminated}{
	\For{i = 1, \dots, k}{
		\If{$A_i$ has not terminated}{
			Get query $q_j^i$ from $A_i$ \;
			Pass answer `no' to $A_i$\; 
			Set $T_i := T_i \cup \{q_j^i\}$\;
		}
		%\If{ $|\vec{q_i}| \ge j$ }{
		%	
		%}
	}
  \For{$\vec{x} \in \prod T_i$}{
  	Query $\vec{x} \in \targ$\;
	\If{$\vec{x} \in \targ$}{
		Run Proposition \ref{memandpos} algorithm using $\vec{x}$ as a positive example\;	
	}
  }
  $j := j+1$\;
}
\caption{Algorithm for Learning from Membership Queries Only}
\end{algorithm}