In the following proofs, we assume we are given concept classes $C_1, C_2, \dots, C_k$ defined over sets $X_1$, $X_2$, \dots, $X_k$. 
Each $c^*_i$ in each $C_i$ is learnable from algorithm $A_i$ (called \emph{sublearners}) using queries to an oracle that can answer any queries in a set $Q$. 
This set $Q$ contains the available types of queries and is a subset of the queries shown in Table \ref{table:queries}.
For example, if $Q := \{ \memQ, \eqQ \}$, then each $A_i$ can make membership and equivalence queries to its corresponding oracle. 
 \todo{Q is almost always a singleton. Should we just call it a query instead of a set?}


For each query $q \in Q$, we say algorithm $A_i$ makes $\genCi{i}$ many $\genQ$ queries to the oracle in order to learn concept $c^*_i$, dropping the index $i$ when necessary .
We replace the term $\genC$ with a more specific term when the type of query is specified.
For example, an algorithm $A$ might make $\memC$ many membership queries to learn $c$. 
\todo{What background information, i.e., explanation of queries, should I include, if any?}


Unless otherwise stated, we will assume any index $i$ or $j$ ranges over the set $\{ 1 \dots k \}$.
We write $\prod S_i$ or $S_1 \times \dots \times S_k$ to refer to the $k$-ary Cartesian product (i.e., cross-product) of sets $S_i$. 
%Note that this is the $k$-ary cartesian product, and is not simply repeated applications of the binary cartesian product.
%So for example $\prod_{i=1}^3 S_i = S_1 \times S_2 \times S_3 = \{ (s_1, s_2, s_3) \mid s_1 \in S_1, s_2 \in S_2, s_3 \in S_3 \}$, which is distinct from $\{ ((s_1, s_2), s_3) \mid s_1 \in S_1, s_2 \in S_2, s_3 \in S_3 \}$.
We use $S^k$ to refer to $\prod_{i=1}^k S$. 

We use vector notation $\vec{x}$ to refer to a vector of elements $(x_1,\dots, x_k)$, $\vec{x}[i]$ to refer to $x_i$, and $\vec{x}[i \leftarrow x'_i]$ to refer to $\vec{x}$ with $x'_i$ replacing value $x_i$ at position $i$. 
We define $\boxtimes^k_{i=1} C_i := \{ \prod c_i \mid c_i \in C_i, i \in \{1,\dots,k\} \}$. \todo{replace products of sets with $\boxtimes$}
We write $\vec{c}$ or $\prod c_i$ for any element of $\boxtimes^k_{i=1} C_i $ and will often denote $\vec{c}$ by $(c_1, \dots, c_k)$ in place of $\prod c_i$. 
The target concept will be represented as $c^*$ or $\targ$ which equals $(c^*_1, \dots, c^*_k)$.

\todo{mention representations}
%We assume that there is a fixed class of representations for any concept class $C$ and use $size(c)$ to denote the number of bits for the representation of $c$.
%The choice of representation class can effect whether a concept class is efficiently learnable.
%In our case, we assume that the 


The results below answer the following question:
\begin{quote}
For different sets of queries $Q$, what is the bound on the number of queries to learn a concept in $\boxtimes C_i $ as a function of each $\genCi{i}$ for each $q \in Q$?
\end{quote}
%For what set of queries $Q$ does the learnability of each $C_i$ imply the learnability of $\prod C_i $ and how does the number of queries to learn $\prod C_i $ increase as a function of each $\genCi{i}$ for each $q \in Q$?

The proofs in this paper make use of the following simple observation:
\begin{observation}
\label{subobs}
For sets $S_1, S_2, \dots, S_k$ and $T_1, T_2, \dots, T_k$, assume $\prod S_i \ne \emptyset$.  Then $\prod S_i \subseteq \prod T_i$ if and only if $S_i \subseteq T_i$, for all $i$.
\end{observation}

