The previous section showed that learning cross products of membership queries requires at most $O(max_i \{ \memCi{i}(c_i) \}^k)$ membership queries. 
A natural next question is whether this can be done for equivalence and subset queries. 
In this section, we answer that question in the negative. 
We will construct a class $\eqhard$ that can be learned from $n$ equivalence or subset queries but which requires at least $k^n$ queries to learn $\eqhard^k$.  

We define $\eqhard$ to be the set $\{ \ntreef(s) \mid s \in \mathbb{N}^* \}$, where $\ntreef(s)$ is defined as follows:

\[\ntreef(\lambda) := \{\lambda\} \times \mathbb{N}\]
\[\ntreef(s) := (\{s\} \times \mathbb{N}) \cup \ntree(s)\]
\[\ntree(sa) := (\{s\} \times (\mathbb{N} \backslash \{a\})) \cup \ntree(s)\]\\

For example, $\ntreef(12) = (\{12\} \times \mathbb{N}) \cup (\{1\}\times(\mathbb{N} \backslash \{2\}))\cup(\{\lambda\} \times (\mathbb{N} \backslash \{1\}))$.


An important part of this construction is that for any two strings $s,s' \in \mathbb{N}$, we have that $\ntreef(s) \subseteq \ntreef(s')$ if and only if $s = s'$. 
This implies that a subset query will return true if and only if the true concept has been found. 
Moreover, an adversarial oracle can always give a negative example for an equivalence query, meaning that oracle can give the same counterexample if a subset query were posed. 
So we will show that $\eqhard$ is learnable from equivalence queries, implying that it is learnable from subset queries. 

We will prove a lower-bound on learning $\eqhard^k$ from subset queries from an adversarial oracle. 
This will imply that $\eqhard^k$ is hard to learn from equivalence queries, since an adversarial equivalence query oracle can give the exact same answers and counterexamples as a subset query oracle.

\begin{proposition}
There exist algorithms for learning from equivalence queries or subset queries such that any concept $\ntreef(s) \in \eqhard$ can be learned from $|s|$ queries. 
\end{proposition}
\begin{proof}\todo{flesh out this proof?}
(sketch) Algorithm \ref{ntree} shows the learning algorithm for equivalence queries, and Figure \ref{singleeqhard} show the decision tree.  
%As mentioned above, this algorithm is essentially the same for learning from subset queries. 
When learning $\ntreef(s)$ for any $s \in \mathbb{N}^*$, the algorithm will construct $s$ by learning at least one new element of $s$ per query. 
Each new query to the oracle is constructed from a string that is a substring of $s$
If a positive counterexample is given, this can only yield a longer substring of $s$.
\end{proof}


\begin{algorithm}[H]
\label{ntree}
\SetAlgoLined
\KwResult{Learns $\eqhard$}
Set $s = \lambda$\;
\While{True}{
	Query $\ntreef(s)$ to Oracle
	\If{Oracle returns `yes'}{
		\Return $\ntreef(s)$
	}
	\If{Oracle returns $(s',m) \in c^* \backslash \ntreef(s)$ }
	{
		Set $s = s'$\;
	}
	\If{Oracle returns $(s,m) \in \ntreef(s) \backslash c^*$}{
		Set $s = sm$\;
	}
} 
\caption{Learning $\eqhard$ from equivalence queries.}
\end{algorithm}

\tikzset{
  treenode/.style = {align=center, inner sep=0pt, text centered,
    font=\sffamily},
    cnode/.style = {treenode, circle, black, font=\sffamily\bfseries, draw=black, text width=2.2em},
    elip/.style = {treenode,  draw=none, black, font=\sffamily\bfseries,  text width=2em},
    edge_style/.style={draw=black}
%  arn_n/.style = {treenode, circle, white, font=\sffamily\bfseries, draw=black,
%    fill=black, text width=1.5em},% arbre rouge noir, noeud noir
%  arn_r/.style = {treenode, circle, red, draw=red, 
%    text width=1.5em, very thick},% arbre rouge noir, noeud rouge
%  arn_x/.style = {treenode, rectangle, draw=black,
%    minimum width=0.5em, minimum height=0.5em}% arbre rouge noir, nil
}

\begin{figure}
\centering
\label{singleeqhard}
\begin{tikzpicture}[->,>=stealth',level/.style={sibling distance = 5cm/#1,
  level distance = 1.5cm}] 
\node[cnode] at (0, 0)   (lam) {$\ntreef(\lambda)$};
\node[cnode] at (-2, -2)   (1) {$\ntreef(1)$};
\node[cnode] at (0, -2)   (2) {$\ntreef(2)$};
\node[elip] at (3, -2)   (elp1) {$...$};
\node[cnode] at (-3.5, -4)   (11) {$\ntreef(11)$};
\node[cnode] at (-2, -4)   (12) {$\ntreef(12)$};
\node[elip] at (1, -4)   (elp2) {$...$};

 \draw[edge_style]  (lam) edge node[above left]{$(\lambda,1)$} (1);
 \draw[edge_style]  (lam) edge node[right]{$(\lambda,2)$} (2);
 \draw[edge_style]  (lam) edge (elp1);
 \draw[edge_style]  (1) edge node[above left]{$(1,1)$} (11);
 \draw[edge_style]  (1) edge node[right]{$(1,2)$} (12);
 \draw[edge_style]  (1) edge (elp2);

%\node [cnode] {}
%    child{ node [cnode] {$\ntreef(1)$}
%            child{ node [cnode] {$\ntreef(11)$} edge from parent node[above left] {$(1, 1)$}}
%            child{ node [cnode] {$\ntreef(12)$} edge from parent node[above right] {$(1, 2)$}}
%              edge from parent node[above left] {$(\lambda, 1)$}                             
%    }
%    child { node [cnode] {$\ntreef(2)$} edge from parent node[above left] {$(\lambda, 2)$}}
%    child { node [cnode] {$\ntreef(2)$} }
%; 
\end{tikzpicture}
\caption{A tree representing Algorithm \ref{ntree}. Nodes are labelled with the queries made at each step, and edges are labelled with the counterexample given by the oracle.}
\end{figure}


\subsection{Showing $\eqhard^k$ is Hard to Learn}

It is easy to learn $\eqhard$, since each new counterexample gives one more element in the target string $s$. 
When learning a concept, $\prod \ntreef(s_i)$, it is not clear which dimension a given counterexample applies to. 
Specifically, a given counterexample $\vec{x}$ could have the property that $\vec{x}[i] \in \ntreef(s_i)$ for all $i \ne j$, but the learner cannot infer the value of this $j$. 
It must then proceed considering all possible values of $j$, requiring exponentially more queries for longer $s_i$. \todo{is this clear?} 
This subsection will formalize this notion to prove an exponential lower bound on learning $\eqhard^k$. 
First, we need a couple definitions. 


A concept $\prod \ntreef(s_i)$ is \emph{justifiable} if one of the following holds:
\begin{itemize}
\item For all $i$, $s_i = \lambda$
\item There is an $i$ and an $a \in \mathbb{N}$ and $w \in \mathbb{N}^*$ such that $s_i = wa$, and the $k$-ary cross-product $\ntreef(s_1) \times \dots \times \ntreef(w) \times \dots \times \ntreef(s_k)$ was justifiably queried to the oracle and received a counterexample $\vec{x}$ such that $\vec{x}[i] = (w, a)$. 
\end{itemize}

A concept is \emph{justifiably queried} if it was queried to the oracle when it was justifiable. 
\newline

For any strings $s,s' \in \mathbb{N}^*$, we write $s \le s'$ if $s$ is a substring of $s'$, and we write $s < s'$ if $s \le s'$ and $s \ne s'$.
We say that the \emph{sum of string lengths} of a concept $\prod \ntreef(s_i)$ is of size $r$ if $\sum |s_i| = r$

Proving that learning is hard in the worst-case can be thought of as a game between learner and oracle. \todo{is this clear?}
The oracle can answer queries without first fixing the target concept. 
It will answer queries so that for any $n$, after less than $k^n$ queries, there is a concept consistent with all given oracle answers that the learning algorithm will not have guessed. 
The specific behavior of the oracle is defined as follows:

\begin{itemize}
\item It will always answer the same query with the same counterexample.  
\item Given any query $\prod \ntreef(s_i) \subseteq c^*$, the oracle will return a counterexample $\vec{x}$ such that for all $i$, $\vec{x}[i] = (s_i, a_i)$, and $a_i$ has not been in any query or counterexample yet seen.
\item The oracle never returns `yes' on any query. 
\end{itemize}

The remainder of this section assumes that queries are answered by the above oracle.
An example of answers by the above oracle and the justifiable queries it yields is given below.

\begin{example}
\label{eqhardex}
Consider the following example when $k = 2$. 
First, the learner queries $(\ntreef(\lambda), \ntreef(\lambda))$ to the oracle and receives a counter-example $((\lambda, 1), (\lambda, 2))$. 
The justifiable concepts are now $(\ntreef(1), \ntreef(\lambda))$ and $(\ntreef(\lambda), \ntreef(2))$. 
The learner queries $(\ntreef(1), \ntreef(\lambda))$ and receives counterexample  $((1, 3), (\lambda, 4))$. 
 The learner queries $(\ntreef(\lambda), \ntreef(2))$ and receives counterexample $((\lambda, 5), (2, 6))$.
The justifiable concepts are now  $(\ntreef(1), \ntreef(4))$, $(\ntreef(1 \cdot 3), \ntreef(\lambda))$, $(\ntreef(5), \ntreef(2))$ and $(\ntreef(\lambda), \ntreef(2 \cdot 6))$.  
At this point, these are the only possible solutions whose sum of string lengths is $2$.
%At this point, the only possible solutions constructible from strings of length $1$ are $(\ntreef(1), \ntreef(4))$ and $(\ntreef(5), \ntreef(2))$. 
The graph of justifiable queries is given in Figure \ref{eqhardtree}.
\end{example}

\tikzset{
  treenode/.style = {align=center, inner sep=0pt, text centered,
    font=\sffamily},
    cnode/.style = {treenode, rectangle, black, font=\sffamily\bfseries, draw=black, text width=9em, text height=1.5em},
    elip/.style = {treenode,  draw=none, black, font=\sffamily\bfseries,  text width=2em},
    edge_style/.style={draw=black}
%  arn_n/.style = {treenode, circle, white, font=\sffamily\bfseries, draw=black,
%    fill=black, text width=1.5em},% arbre rouge noir, noeud noir
%  arn_r/.style = {treenode, circle, red, draw=red, 
%    text width=1.5em, very thick},% arbre rouge noir, noeud rouge
%  arn_x/.style = {treenode, rectangle, draw=black,
%    minimum width=0.5em, minimum height=0.5em}% arbre rouge noir, nil
}

\begin{figure}
\label{eqhardtree}
\begin{tikzpicture}[->,>=stealth',level/.style={sibling distance = 5cm/#1,
  level distance = 1.5cm}] 
\node[cnode] at (-.5, 0)   (0) {JQ: $(\ntreef(\lambda),\ntreef(\lambda))$ \vspace{1mm} \\ CE: $((\lambda, 1),(\lambda, 2)) $ \vspace{2mm}   };

\node[cnode] at (-4, -2)   (10) {JQ: $(\ntreef(1),\ntreef(\lambda))$ \vspace{1mm} \\ CE: $((1, 3),(\lambda, 4))$ \vspace{2mm} };
\node[cnode] at (3.4, -2)   (11) {JQ: $(\ntreef(\lambda),\ntreef(2))$ \vspace{1mm} \\ CE: $((\lambda, 5),(2,6))$ \vspace{1mm} };

\node[cnode] at (-6, -4)   (20) {JQ: $(\ntreef(1\cdot 3),\ntreef(\lambda))$ \vspace{2mm} };
\node[cnode] at (-2.25, -4)   (21) {JQ: $(\ntreef(1),\ntreef(4))$ \vspace{2mm}};
\node[cnode] at (1.5, -4)   (22) {JQ: $(\ntreef(5),\ntreef(2))$ \vspace{2mm}};
\node[cnode] at (5.25, -4)   (23) {JQ: $(\ntreef(\lambda),\ntreef(2\cdot 6))$ \vspace{2mm}};

 \draw[edge_style]  (0) edge node[above left]{$1 \le s_1$} (10);
 \draw[edge_style]  (0) edge node[above right]{$2 \le s_2$} (11);
 
  \draw[edge_style]  (10) edge node[left]{$1,3 \le s_1$} (20);
 \draw[edge_style]  (10) edge node[right]{$4 \le s_2$} (21);
  \draw[edge_style]  (11) edge node[left]{$5 \le s_1$} (22);
 \draw[edge_style]  (11) edge node[right]{$2, 6 \le s_2$} (23);
 %\draw[edge_style]  (0) edge (elp1);
 %\draw[edge_style]  (1) edge node[above left]{$(1,1)$} (11);
 %\draw[edge_style]  (1) edge node[right]{$(1,2)$} (12);
 %\draw[edge_style]  (1) edge (elp2);

%\node [cnode] {}
%    child{ node [cnode] {$\ntreef(1)$}
%            child{ node [cnode] {$\ntreef(11)$} edge from parent node[above left] {$(1, 1)$}}
%            child{ node [cnode] {$\ntreef(12)$} edge from parent node[above right] {$(1, 2)$}}
%              edge from parent node[above left] {$(\lambda, 1)$}                             
%    }
%    child { node [cnode] {$\ntreef(2)$} edge from parent node[above left] {$(\lambda, 2)$}}
%    child { node [cnode] {$\ntreef(2)$} }
%; 
\end{tikzpicture}
\caption{
The tree of justifiable queries used in Example \ref{eqhardex}. 
Each node lists the justifiable query (JQ) and counterexample (CE) given for that query. 
The edges below each node are labelled with the possible inferences about $s_1$ and $s_2$ that can be drawn from the counterexample.
}
\end{figure}
\todo{fix figure caption and box/text formatting}



The following simple proposition can be proven by induction on sum of string lengths.

\begin{proposition}
\label{subjust}
Let $\prod \ntreef(s_i)$ be a justifiable concept. 
Then for all $w_1$, $w_2$, \dots, $w_k$ where for all $i$, $w_i \le s_i$, $\prod \ntreef(w_i)$ has been queried to the oracle.
\end{proposition}

\begin{proposition}
\label{numjustconc}
If all justified concepts $\prod \ntreef(s_i)$ with sum of string lengths equal to $r$ have been queried, then there are $k^{r+1}$ justified queries whose sum of string lengths equals $r+1$
\end{proposition}
\begin{proof}
This proof follows by induction on $r$. 
When $r=0$, the concept $\prod \ntreef(\lambda)$ is justifiable.%has been queried and a counterexample $\vec{x}$ has been given.
%Then for all $i$, the concept which is $\ntreef(\lambda)$ at all $j \ne i$ and $\ntreef(\vec{x}[i])$ is justifiable (there are $k$ such $i$.
For induction, assume that there are $k^r$ justifiable queries with sum of string lengths equal to $r$. 
By construction, the oracle will always chose counterexamples with as-yet unseen values in $\mathbb{N}$. 
So querying each concept $\prod \ntreef(s_i)$ will yield a counterexample $\vec{x}$ where for all $i$, $\vec{x}[i] = (s_i, a_i)$ for new $a_i$.
Then for all $i$, this query creates the justifiable concept $\prod \ntreef(s'_j)$, where $s'_j = s_j$ for all $j \ne i$ and $s'_i = \ntreef(s_i \cdot a_i)$.
Thus there are $k^{r+1}$ justifiable concepts with sum of string lengths equal to $r+1$.
\end{proof}

We are finally ready to prove the main theorem of this section.

\begin{theorem}
Any algorithm learning $\eqhard^k$ from subset (or equivalence) queries requires at least $k^r$ queries to learn a concept $\prod \ntreef(s_i)$, whose sum of string lengths is $r$.
Equivalently, the algorithm takes $k^{\sum \genCi{i}}$ subset (or equivalence) queries.
\end{theorem}
\begin{proof}\todo{Should I go into more detail why the existence of this c' shows the algorithm hasn't learned c?}
Assume for contradiction that an algorithm can learn with less than $k^r$ queries and let this algorithm converge on some concept $c = \prod \ntreef(s_i)$ after less than $k^r$ queries.
%We will construct another concept $c' = \prod \ntreef(s'_i)$ that is consistent with all given oracle answers, but whose sum of string lengths is less than $r$.
Since less than $k^r$ queries were made to learn $c$, by Proposition \ref{numjustconc}, there must be some justifiable concept $c' = \prod \ntreef(s'_i)$ with sum of string lengths less than or equal to $r$ that has not yet been queried.
By Proposition \ref{subjust}, we can assume without loss of generality that for all $w_i \le s_i'$, $\prod \ntreef(w_i)$ has been queried to the oracle.
We will show that $c'$ is consistent with all given oracle answers, contradicting the claim that $c$ is the correct concept. 
Let $c_v := \prod \ntreef(v_i)$ be any concept queried to the oracle, and let $\vec{x}$ be the given counterexample.
If for all $i$, $v_i \le s'_i$, then by construction, there is a $j$ with $\vec{x}[j] = (v_j, a_j)$ such that $v_j \cdot a_j \le s'_j$, so $\vec{x}$ is a valid counterexample.
Otherwise, there is an $i$ such that $v_i \not\le s'_i$. 
So $(\{v_i\} \times \mathbb{N})  \cap \ntreef(s'_i) = \emptyset$, so $\vec{x}$ is a valid counterexample. 
Therefore, all counterexamples are consistent with $c'$ being correct concept, contradicting the claim that the learner has learned $c$. 
\end{proof}







