This sections discusses the problem of PAC-learning the cross products of concept classes. \todo{do we want to assume familiarity with VC dimension?}

Previously, van Der Vaart and Weller \cite{van2009note} have shown the following bound on the VC-dimension of cross-products of sets:
\[\VC(\prod C_i) \le a_1 log(k a_2) \sum \VC(C_i)\] 

Here $a_1$ and $a_2$ are constants with $a_1 \approx 2.28$ and $a_2 \approx 3.92$.
As always, $k$ is the number of concept classes included in the cross-product.

The VC-dimension gives a bound on the number of labelled examples needed to PAC-learn a concept, but says nothing of the computational complexity of the learning process. 
This complexity mostly comes from the problem of finding a concept in a concept class that is consistent with a set of labelled examples. 
We will show that the complexity of learning cross-products of concept classes is a polynomial function of the complexity of learning from each individual concept class.

First, we will describe some necessary background information on PAC-learning.

\subsection{PAC-learning Background}






\begin{definition}
Let $C$ be a concept class over a space $X$. 
We say that $C$ is efficiently PAC-learnable if there exists an algorithm $A$ with the following property:
For every distribution $\dist$ on $X$, every $c \in C$, and every $\epsilon, \delta \in (0, 1)$,
if algorithm $A$ is given access to $EX(c,\dist)$ then with probability $1 - \delta$, $A$ will return a $c' \in h$ such that $error(c') \le \epsilon$. 
$A$ must run in time polynomial in $1/\epsilon$, $1/\delta$, and $size(c)$. 
\end{definition}

We will refer to $\epsilon$ as the `accuracy' parameter and $\delta$ as the `confidence' parameter.
The value of $error(c)$ is the probability that for an $x$ sampled from $\dist$ that $c(x) \ne c^*(x)$. 
 PAC-learners have a \emph{sample complexity} function $m_C(\epsilon, \delta): (0,1)^2 \rightarrow \mathbb{N}$. 
The sample complexity is the number of samples an algorithm must see in order to probably approximately learn a concept with parameters $\epsilon$ and $\delta$.


Given a set $S$ of labelled examples in $X$, we will use $A(S)$ to denote the the concept class the algorithm $A$ returns after seeing set $S$ \todo{check whether we should use S as an input or as something given in alg}.


A learner $A$ is an \emph{empirical risk minimizer} if $A(C)$ returns a $c \in C$ that minimizes the number of misclassified examples (i.e., it minimizes $|\{(x,b) \in S \mid c^*(x) \ne b\}|$). 

Empirical risk minimizers are closely related to VC dimension and PAC-learnability as shown in the following theorem (Theorem 6.7 from \cite{shalev2014understanding})

\begin{theorem}
\label{fundpac}
If the concept class $C$ has VC dimension $d$, then there is a constant, $b$, such that applying an Empirical Risk Minimizer $A$ to $m_C(\epsilon, \delta)$ samples will PAC-learn in $C$, where
\[m_C(\epsilon, \delta) \le b \frac{d \cdot log(1/\epsilon) + log(1/\delta)}{\epsilon} \]
\end{theorem}

Finally, we will discuss the \emph{growth function}. 
The growth function describes how many distinct assignments a concept class can make to a given set of elements.  
More formally, for a concept class $C$ and $m \in \mathbb{N}$, the growth function $G_C(m)$ is defined by:
\[ G_C(m) = \max_{x_1,x_2,\dots,x_m} \abs[\Big]{\{ (c(x_1), c(x_2), \dots, c(x_m)) \mid c \in C\} }\]

Each $x_i$ in the above equation is taken over all possible elements of $X_i$.
The VC-dimension of a class $C$ is the largest number $d$ such that $G_C(d) = 2^d$. 

We will use the following bound, a corollary of the Perles-Sauer-Shelah Lemma, to bound the runtime of learning cross-products \cite{shalev2014understanding}. 

\begin{lemma}
\label{pss}
For any concept class $C$ with VC-dimension $d$ and $m > d+1$:
\[ G_C(m) \le (em/d)^d\]
\end{lemma}

\subsection{PAC-Learning Cross-Products}

We now have enough background to describe the strategy for PAC-learning cross-products.
We will just describe learning the cross-product of two concepts. \todo{do we want to describe the general strategy?} 
As above, assume concept classes $C_1$ and $C_2$ and PAC-learners $A_1$ and $A_2$ are given.
We define $A_i(\epsilon, \delta)$ as the runtime of the sublearner $A_i$ to PAC-learn with accuracy parameter $\epsilon$ and confidence parameter $\delta$. 

Assume that $C_1$ and $C_2$ have VC-dimension $d_1$ and $d_2$, respectively. 
We can use the bound from van Der Vaart and Weller to get an upper bound $d$ on the VC-dimension of their cross-product.
Assume the algorithm is given an $\epsilon$ and $\delta$ and there is a fixed target concept $c^* = c_1^* \times c_2^*$.
Theorem \ref{fundpac} gives a bound on the sample complexity $m_{C_1 \times C_2}(\epsilon, \delta)$. 
The algorithm will take a sample of labelled examples of size $m_{C_1 \times C_2}(\epsilon, \delta)$. 
Our goal is to construct an Empirical Risk Minimizer for $C_1 \times C_2$. 
In our case, $c_1 \in C_1$ and $c_2 \in C_2$.
Therefore,  for any sample $S$, an Empirical Risk Minimizer will yield a concept in $C_1 \times C_2$ that is consistent with $S$. 
This algorithm is show in Algorithm \ref{recpaclearn}.


%%%%%Below Here
%\begin{comment}
So let $S$ be any such sample the algorithm takes. 
This set can easily be split into positive examples $S^+$ and negative examples $S^-$, both in $X_1 \times X_2$.
The algorithm works by maintaining sets labeled samples $L_1$ and $L_2$ for each dimension. 
For any $(x_1, x_2) \in S^+$, it holds that $x_1 \in c^*_1$ and $x_2 \in c^*_2$ so $(x_1, \top)$ and $(x_2, \top)$ are added to $L_1$ and $L_2$ respectively. 
For any $(x_1, x_2) \in S^-$, we know that $x_1 \not\in c^*_1$ or $x_2 \not\in c^*_2$ (or both), but it is not clear which is true. 
However, since the goal is only to create an Empirical Risk Minimizer, it is enough to find any concepts $C_1$ and $C_2$ that are consistent with these samples. 
In other words, we need to find a $c_1 \in C_1$ and a $c_2 \in C_2$ such that for every $(x_1, x_2) \in S^+$, $x_1 \in c_1$ and $x_2 \in c_2$ and for all $(x_1, x_2) \in S^-$, either $x_1 \not\in c^*_1$ or $x_2 \not\in c^*_2$.
One idea would be to try out all possible assignments to elements in $S^-$ and check if any such assignment fits any possible concepts. \todo{is it clear what "all possible assignments" are?}
This, however, would be exponential in $|S^-|$. 

Bounding the size of the growth function can narrow this search. 
Specifically, let $S_1^- := \{ x \mid \exists y, (x,y) \in S^-\}$, let $m = |S_1^-|$ and order the elements of $S_1^-$ by $x_1, x_2, \dots, x_m$. % and $S_2^- = \{ x \mid \exists y, (y,x) \in S^-\}$. 
By the definition of the growth function and Lemma \ref{pss}:
\[ |\{ (c(x_1), c(x_2), \dots, c(x_m)) \mid c \in C_1\}| \le G_{C_1}(m) \le (em/d)^d \]
In other words, there are less than $(em/d)^d$ assignments of truth values to elements of $S_1^-$ that are consistent with some concept in $C_1$.
If the algorithm can check every $c_1 \in C_1$ consistent with $S^+$ and $S^-_1$, it can then call $A_2$ to see if there is any $c_2 \in C_2$ such that $(c_1 \times c_2)$ assigns true to every element in $S^+$ and false to every element in $S^-$. \todo{be clear about difference between a set of pairs of labeled examples (such as $S^-$) and one side of that set (such as $S^-_1$)} 

Finding these consistent elements of $C_1$ is made easier by the fact that we can check whether partial assignments to $S^-_1$ are consistent with any concept in $C_1$. 
As mentioned above, it starts by creating the sets $L_1$ and $L_2$ containing all samples in the first and second dimension of $S^+$, respectively.
It then iteratively adds labeled samples from $S^-$.
At each step, the algorithm chooses one element $(x_1, x_2) \in S^-$ at a time and checks which possible assignments to $x_1$ are consistent with $L_1$.
If $(x_1, \bot)$ is consistent, it adds $(x_1, \bot)$ to $L_1$ and calls $RecursiveFindSubconcepts$ on $L_1$ and $L_2$.
 \todo{just add a description saying a sample is consistent with a class when theres a concept that fits the sample}
If $(x_1, \top)$ is consistent with $C_1$, then the algorithm adds $(x_1, \top)$ to $L_1$ and $(x_2, \bot)$ to $L_2$ and calls $RecursiveFindSubconcepts$.
In either case, if an assignment is not consistent, no recursive call is made. 

%To efficiently find the elements of $C_1$ consistent with $S^+$ and $S^-_1$ can be accomplished by something like binary search, as shown in Algorithm \ref{recpaclearn}.



\begin{theorem}
%Let $C_1$ and $C_2$ be concept classes and let $A_1$ and $A_2$ be PAC-learners with runtime $A_1(\epsilon, \delta)$ and $A_2(\epsilon, \delta)$, respectively. 
Let concept classes $C_1$ and $C_2$ have VC-dimension $d_1$ and $d_2$, respectively.
There exists a PAC-learner for $C_1 \times C_2$ that can learn any concept using a sample of size $m = ((d_1 + d_2) \cdot log(1/\epsilon) + log(1/\delta))/\epsilon$.\todo{there are some constants in there}
The learner requires time $O(m^{d_1}(A_1(1/m, log(\delta)) +  A_2(1/m, log(\delta))))$.
\end{theorem}

\todo{testing}

\iffalse
\subsection{Efficient PAC-learning with Membership Queries}

Although polynomial, the complexity of PAC-learning cross-products from a $\pacQ$ oracle is fairly expensive. 
We will show that when a learner is allowed to make membership queries, PAC-learning cross-products becomes much more efficient. 
This is due to the previously shown technique, which uses membership queries and a single positive example to determine on which dimensions a negatively labelled example fails. 

In this case, assuming that $\emptyset \in \boxtimes C_i$, we can ignore the assumption that a positive example is given. 
If no positive example appears in a large enough labeled sample, the the algorithm can pose $\emptyset$ as the hypothesis.\todo{show that this pac-learns?}



%The algorithm takes a sample $S$ of size $m_C(\epsilon, \delta)$ and uses any positive example $\vec{p} \in S$. 
%If $S$ contains no positive example, then the algorithm can pose $\emptyset$ as the hypothesis. \todo{show that this pac-learns?}



If $S$ does contain a positive example $\vec{p}$, then $S$ can be broken down into labeled samples for each dimension $i$. 
The algorithm initialize the sets of positive and negative examples to $S^+_i := \{ \vec{x}[i] \mid  (\vec{x}, \top) \in S\}$ and $S^-_i := \{ \}$, respectively.
For each $(\vec{x}, \bot) \in S$, a membership queries $\vec{p}[i \leftarrow \vec{x}[i]] \in \targ$. 
If so, $\vec{x}[i] $ is added to $S^+_i$.
Otherwise it is added to $S^-_i$.
This labelling is correct by Observation \ref{posobs}.
The set of labelled examples $S_i := (S_i^+ \times \{\top\}) \cup (S_i^- \times \{\bot\})$ is then passed to the sublearner $A_i$.
$A_i$ is run on $S_i$ with accuracy parameter $\epsilon' := \epsilon / k$ and confidence parameter $\delta' := \delta / k$ \todo{DoubleCHECK THIS}.


\begin{proposition}
%Assume each sublearner has runtime $A_i(\epsilon, \delta)$ for accuracy and confidence parameters $\epsilon$ and $\delta$.
The algorithm described above PAC-learns from the concept class $\boxtimes C_i$ with accuracy $\epsilon$ and confidence $\delta$.
It makes $m_C(\epsilon, \delta)$ queries to $\pacQ$, $k \cdot m_C(\epsilon, \delta)$ membership queries, and has runtime $O(\sum A_i(\epsilon / k, \delta / k))$.
\end{proposition}
\begin{proof}
After running the algorithm 
\end{proof}

\fi
%Empirical Risk Minimization



%\begin{algorithm}[H]
%\label{recpaclearn}
%\SetAlgoLined
%\KwResult{Find Subconcepts Consistent with Sample}
%%\SetKwFunction{FMain}{FindSubconcepts}{}
%\SetKwProg{FMain}{FindSubconcepts}{}{}
%\SetKwProg{FRec}{RecursiveFindSubconcepts}{}{}
%
%
%\KwIn{
%$S^+$: Set of positive examples in $X_1 \times X_2$\\
%$S^-$: Set of negative examples in $X_1 \times X_2$\\
%$\delta:$ Confidence parameter in $(0,1)$
%}
% \FMain{($S^+$, $S^-$, $\delta$)}{
% $\delta' := \delta / (|S^-|G_{C_1}(|S^-|) + G_{C_2}(|S^-|))$\;\todo{describe this parameter}
% $\epsilon' := 1 / |S|$\;
% \tcp{$L_1$: Labelled samples in $X_1$}
% $L_1 := \{(x_1, \top) \mid \exists y,  (x_1, y) \in S^+ \}$\;
%  \tcp{$L_2$: Labelled samples in $X_2$}
% $L_2 := \{(x_2, \top) \mid \exists y,  (y, x_2) \in S^+ \}$\; 
% $U := S^-$\;
% \Return RecursiveFindSubconcepts($L_1$, $L_2$, $U$, $\epsilon'$, $\delta'$) \;
% }
%
%
%%\KwIn{
%%$L_1$: Set of labelled samples in $X_1 \times \{\bot, \top\}$\\
%%$L_2$: Set of labelled samples in $X_2 \times \{\bot, \top\}$\\
%%$U$: Set of `unlabelled' samples (known to be negative) in $(X_1 \times X_2)$\\
%%$\epsilon', \delta' \in (0,1)$
%%}
%\FRec{($L_1$, $L_2$, $U$, $\epsilon'$, $\delta'$):}{
%
%\tcp{Run once all labels are assigned to the first dimension of $U$}
%\tcp{Attempts to find concept in $C_2$ consistent with all labels given in $L_2$}
%\If{$U = \emptyset$}{
%	\eIf{$A_2(L_2, \epsilon', \delta')$}{
%		\Return{$(A_1(L_1, \epsilon', \delta'), A_2(L_2, \epsilon', \delta'))$}\;
%	}{$\bot$\;}	
%}
%
%Get $(x_1, x_2) \in U$\;
%$U := U \backslash \{(x_1, x_2)\} $\;
%
% \tcp{Attempts to label $x_1$ as false in $c^*_1$}
%\If{$A_1(L_1 \cup \{ (x_1, \bot) \}, \epsilon', \delta') \ne \bot$}{
%	$c = RecursiveFindSubconcepts(L_1 \cup \{ (x_1, \bot) \}, L_2, U, \epsilon', \delta')$\;
%	\If{$c \ne \bot$}{\Return c\;}
%}
%
%\tcp{Attempts to label $x_1$ as true in $c^*_1$}
%\If{$A_1(L_1 \cup \{ (x_1, \top) \}, \epsilon', \delta') \ne \bot$}{
%	$c = RecursiveFindSubconcepts(L_1 \cup \{ (x_1, \top) \}, L_2 \cup \{(x_2, \bot) \}, U, \epsilon', \delta')$\;
%	\If{$c \ne \bot$}{\Return c \;}
%}
%
%}
%
%\caption{Learning Disjoint Unions}
%\end{algorithm}
%\end{comment}
