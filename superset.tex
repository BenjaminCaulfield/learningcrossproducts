This section introduces arguably the simplest positive result of the paper: when using superset queries, learning cross-products of concepts is as easy as learning the individual concepts. 

Like all positive results in this paper, this is accomplished by algorithm that takes an oracle for the cross-product concept $\prod c_i^*$ and simulates the learning process for each sublearner $A_i$ by acting as an oracle for each such sublearner. 

\todo{make sure subconcept sublearner, etc, is defined}

\todo{should I give more intuition? It's all in the proof fwiw}

%For any set $

%As this is the simplest such reduction, it is worth understanding this 

%As shown in the introduction, it is easy to handle positive examples when learning cross-products. 
%For any $(x_1, \dots, x_k) \in \prod c_i^*$, it holds that $x_i \in c_i^*$ for all $i$. 
%Learning from superset queries is easy since all counter-examples are positive. 
%So for each superset query $\prod S_i$, either the oracle d


\begin{proposition}
%Assume for each concept class $C_i$ there is a learning algorithm $A_i$ that learns each $c_i \in C_i$ in $$ superset queries (we don't assume $Q_i(c_i)$ is finite for all $c_i$). 

If $Q = \{ \supQ \}$, then there is an algorithm that learns any concept $\targ \in \prod C_i $ in $\sum \supCi{i}(c^*_i)$ queries.  
\end{proposition}
\begin{proof}
Algorithm \ref{supalg} learns $\Boxtimes C_i $ by simulating the learning of each $A_i$ on its respective class $C_i$. 
The algorithm asks each $A_i$ for superset queries $S_i \supseteq c_i^*$, queries the product $\prod S_i$ to the oracle, and then uses the answer to answer at least one query to some $A_i$. \todo{should there be a special symbol for queries instead of just $\supseteq$}
Since at least one $A_i$ receives an answer for each oracle query, at most $\sum \supCi{i}(c^*_i)$ queries must be made in total. 

 

We will now show that each oracle query results in at least one answer to an $A_i$ query (and that the answer is correct). 
The oracle first checks if the target concept is empty and stops if so.
If no concept class contains the empty concept, this check can be skipped. 
At each step, the algorithm poses query $\prod S_i$ to the oracle. 
If the oracle returns 'yes' (meaning $\prod S_i \supseteq \targ$), then  $S_i \supseteq c_i^*$ for each $i$ by Observation \ref{subobs}, so the oracle answers 'yes' to each $A_i$. 
If the oracle returns 'no', it will give a counterexample $\vec{x} = (x_1,\dots,x_k) \in \targ \backslash \prod S_i$. 
There must be at least one $x_i \not\in S_i$ (otherwise, $\vec{x}$ would be in $\prod S_i$). 
So the algorithm checks $x_j \in S_j$ for all $x_j$ until an $x_i \not\in S_i$ is found. 
Since $\vec{x} \in \targ$, we know $x_i \in c_i^*$, so $x_i \in c_i^* \backslash S_i$, so the oracle can pass $x_i$ as a counterexample to $A_i$. 

Note that once $A_i$ has output a correct hypothesis $c_i$, $S_i$ will always equal $c_i$, so counterexamples must be taken from some $j \ne i$. 
\end{proof}





\begin{algorithm}[H]
\label{supalg}
\SetAlgoLined
\KwResult{Learn $\prod C_i $ from Superset Queries}
\If{$\emptyset \in C_i$ for some $i$}{
	Query $\emptyset \supseteq \targ$\;
	\If{$\emptyset \supseteq \targ$}{
		\Return{$\emptyset$}
	}
}
\For{$i = 1 \dots k$}{
	Set $S_i$ to initial subset query from $A_i$
}
 \While{Some $A_i$ has not completed}{
  Query $\prod S_i$ to oracle\;
  \eIf{$\prod S_i \supseteq c^*$ }{
   Answer $S_i \supseteq c_i^*$ to each $A_i$\;
   Update each $S_i$ to new query\;
   }{
 	Get counterexample $\vec{x} = (x_1,\dots,x_k)$
   	\For{i = 1 \dots k}{
   		\If{$x_i \not\in S_i$}{
			Pass counterexample $x_i$ to $A_i$\;
			Update $S_i$ to new query\;
			} 
  		}
 	}
	\For{i = 1 \dots k}{
		\If{$A_i$ outputs $c_i$}{
			Set $S_i := c_i$\;
		}
	}
 }
 \Return{$\prod c_i$}\;
 \caption{Algorithm for learning from Subset Queries}
\end{algorithm}
