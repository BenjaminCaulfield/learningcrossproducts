The final collection of query complexities is given in Figure \ref{complexitytable}.
All of the bounds are tight, except for learning with superset queries, membership queries, and one positive example.  

\begin{figure}
\begin{center}
\renewcommand{\arraystretch}{1.5}
\begin{tabular}{ |c|c|c|c| } 
\cline{2-4}
\multicolumn{1}{c|}{} & Only $Q$ & \multicolumn{2}{c|}{$Q$ with $\memQ$ and $\oneposQ$} \\
\cline{1-1}
\multicolumn{1}{|c|}{$Q \downarrow$} & \genC & \memC & \genC \\
\hline
\posQ & Not Possible  & Not Possible & Not Possible \\
\hline
\supQ & $\sum \supCi{i}$ & $0$ & $\sum \supCi{i}$\\
\hline
\memQ & $(max_i\{\memCi{i}\})^k$ & $\sum \memCi{i}$ & $\sum \memCi{i}$ \\
\hline
\subQ & $k^{\sum \subCi{i}}$ & $k \sum \subCi{i}$  & $\sum \subCi{i}$ \\
\hline
\eqQ  &$k^{\sum \eqCi{i}}$ &  $k \sum \eqCi{i}$ &  $\sum \eqCi{i}$\\
\hline
\end{tabular}
\renewcommand{\arraystretch}{1}
\end{center}
\caption{
Final collection of query complexities. 
The rows represents the set $Q$ of queries needed to learn each $C_i$.  
The columns determine whether the cross product is learned from queries in just $Q$ or $Q \cup \{\memQ, \oneposQ\}$. 
In the latter case, the column is separated to track the number of membership queries and queries in $Q$ that are needed.
 }
 \label{complexitytable}
\end{figure}