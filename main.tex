\documentclass[11pt]{amsart}
\usepackage{geometry}                % See geometry.pdf to learn the layout options. There are lots.
\geometry{letterpaper}                   % ... or a4paper or a5paper or ... 
%\geometry{landscape}                % Activate for for rotated page geometry
%\usepackage[parfill]{parskip}    % Activate to begin paragraphs with an empty line rather than an indent
\usepackage{graphicx}
\usepackage{amssymb}
\usepackage{epstopdf}
\usepackage{algorithm2e}
\usepackage{tabularx}
\usepackage{todonotes}

\DeclareGraphicsRule{.tif}{png}{.png}{`convert #1 `dirname #1`/`basename #1 .tif`.png}
%\documentclass[11pt]{amsart}
\usepackage{geometry}                % See geometry.pdf to learn the layout options. There are lots.
\geometry{letterpaper}                   % ... or a4paper or a5paper or ... 
%\geometry{landscape}                % Activate for for rotated page geometry
%\usepackage[parfill]{parskip}    % Activate to begin paragraphs with an empty line rather than an indent
\usepackage{graphicx}
\usepackage{amssymb}
\usepackage{epstopdf}
\usepackage[noend]{algorithm2e}
\usepackage{tabularx}
\usepackage[disable]{todonotes}
%\usepackage{todonotes}
\usepackage{multirow}

\usepackage{subfig}
\usepackage{mathtools}

\DeclarePairedDelimiter\abs{\lvert}{\rvert}

%\usepackage{authblk}

%\usepackage{slashbox}
\usepackage{ amssymb }
\usepackage{tikz}
\usetikzlibrary{arrows}
\usepackage{varwidth}
\usepackage{listings}
\lstset{
  basicstyle=\ttfamily,
  mathescape
}

\newcommand\mycommfont[1]{\footnotesize\ttfamily\textcolor{gray}{#1}}
\SetCommentSty{mycommfont}


\DeclareGraphicsRule{.tif}{png}{.png}{`convert #1 `dirname #1`/`basename #1 .tif`.png}

%\newtheorem{theorem}{Theorem}
%\newtheorem{lemma}{Lemma}
%\newtheorem{proposition}{Proposition}
%\newtheorem{remark}{Remark}
\newtheorem{observation}{Observation}
%\newtheorem{definition}{Definition}

%\theoremstyle{definition}
%\newtheorem{example}{Example}

\newcommand{\targ}{\prod c^*_i}
\newcommand{\dist}{\mathcal{D}}

\newcommand{\genQ}{q}
\newcommand{\oneposQ}{\emph{1}Pos}
\newcommand{\posQ}{Pos}
\newcommand{\memQ}{\emph{Mem}}
\newcommand{\subQ}{\emph{Sub}}
\newcommand{\supQ}{\emph{Sup}}
\newcommand{\eqQ}{\emph{EQ}}
\newcommand{\pacQ}{\emph{EX}}

\newcommand{\genQi}[1]{q_{#1}}
\newcommand{\oneposQi}[1]{\emph{1}Pos_{#1}}
\newcommand{\posQi}[1]{Pos_{#1}}
\newcommand{\memQi}[1]{\emph{Mem}_{#1}}
\newcommand{\subQi}[1]{\emph{Sub}_{#1}}
\newcommand{\supQi}[1]{\emph{Sup}}
\newcommand{\eqQi}[1]{\emph{EQ}_{#1}}
\newcommand{\pacQi}[1]{\emph{EX}_{#1}}

%\newcommand{\oneposQ}{$1$-pos}
%\newcommand{\memQ}{\emph{Mem}}
%\newcommand{\subQ}{\emph{Sub}}
%\newcommand{\supQ}{\emph{Sup}}
%\newcommand{\eqQ}{\emph{EQ}}
%\newcommand{\pacQ}{\emph{PAC}}


%Complexity of Queries
\newcommand{\genC}{\#q}
\newcommand{\oneposC}{\#$1$Pos}
\newcommand{\posC}{\#Pos}
\newcommand{\memC}{\#\emph{Mem}}
\newcommand{\subC}{\#\emph{Sub}}
\newcommand{\supC}{\#\emph{Sup}}
\newcommand{\eqC}{\#\emph{EQ}}
\newcommand{\pacC}{\#\emph{EX}}

\newcommand{\genCi}[1]{\#q_{#1}}
\newcommand{\oneposCi}[1]{\#$1$Pos_{#1}}
\newcommand{\posCi}[1]{\#Pos_{#1}}
\newcommand{\memCi}[1]{\#\emph{Mem}_{#1}}
\newcommand{\subCi}[1]{\#\emph{Sub}_{#1}}
\newcommand{\supCi}[1]{\#\emph{Sup}}
\newcommand{\eqCi}[1]{\#\emph{EQ}_{#1}}
\newcommand{\pacCi}[1]{\#\emph{EX}_{#1}}



%EQ and Sub are Hard construction
\newcommand{\ntreef}{\mathfrak{c}}
\newcommand{\ntree}{\mathfrak{c}_{sub}}
\newcommand{\eqhard}{\mathfrak{C}}


\newcommand{\VC}{\mathcal{V}\mathcal{C}}



%\newcommand{\xsubQ}{\emph{XSub}}
%\newcommand{\xsupQ}{\emph{XSup}}
%\newcommand{\xeqQ}{\emph{XEQ}}
%\newcommand{\ressubQ}{\emph{resSub}}
%\newcommand{\ressupQ}{\emph{resSup}}
%\newcommand{\reseqQ}{\emph{resEQ}}




\makeatletter
\def\moverlay{\mathpalette\mov@rlay}
\def\mov@rlay#1#2{\leavevmode\vtop{%
   \baselineskip\z@skip \lineskiplimit-\maxdimen
   \ialign{\hfil$\m@th#1##$\hfil\cr#2\crcr}}}
\newcommand{\charfusion}[3][\mathord]{
    #1{\ifx#1\mathop\vphantom{#2}\fi
        \mathpalette\mov@rlay{#2\cr#3}
      }
    \ifx#1\mathop\expandafter\displaylimits\fi}
\makeatother

\newcommand{\cupdot}{\charfusion[\mathbin]{\cup}{\cdot}}
\newcommand{\bigcupdot}{\charfusion[\mathop]{\bigcup}{\cdot}}



\newcommand{\disClass}{C_{\cupdot}}


\title{Learning Products of Learnable Sets}
\author{}
%\date{}                                           % Activate to display a given date or no date

\begin{document}
\maketitle
%\section{}
%\subsection{}

\begin{table}
\begin{center}
  \begin{tabularx}{\textwidth}{| c | c | c | X | }
    \hline
    Query Name & Symbol & Complexity & Oracle Definition \\ \hline
    Single Positive Query & $\oneposQ$ & $\oneposC(c)$ & Return a fixed $x \in c^*$ \\ \hline
    Positive Query & $\posQ$ & $\posC(c)$ & Return an $x\in c^*$ that has not yet been given as a positive example (if one exists)\\ \hline
    Membership Query & $\memQ$ & $\memC(c)$ & Given string $s$, return true iff $s \in c^*$ \\ \hline
    Equivalence Query & $\eqQ$ & $\eqC$ & Given $c \in C$, return true if $c=c^*$ otherwise return $x \in (c \backslash c^*) \cup (c^* \backslash c)$\\ \hline 
    Subset Query & $\subQ$ & $\subC(c)$ & Given $c \in C$, return `true' if $c \subseteq c^*$ \mbox{  } otherwise return some $x \in c \backslash c^*$ \\ \hline
    Superset Query & $\supQ$ & $\supC(c)$ & Given $c \in C$, return `true' if $c \supseteq c^*$  otherwise return some $x \in c^* \backslash c$\\ \hline
  \end{tabularx}
\end{center}
\end{table}

\section{Important Notation}

In the following proofs, we assume we are given concept classes $C_1, C_2, \dots, C_k$ defined over sets $X_1$, $X_2$, \dots, $X_k$. 
Each $c_i$ in each $C_i$ is learnable from algorithm $A_i$ using queries to an oracle that can answer any queries in a set $Q$. 
For each query $q \in Q$, we say algorithm $A_i$ makes $\#q_i(c)$ many $q$ queries to the oracle in order to learn concept $c$, dropping the index $i$ when necessary .
For example, an algorithm $A$ might make $\memC(c)$ many membership queries to learn $c$. 

Unless otherwise stated, we will assume any index $i$ or $j$ ranges over the set $\{ 1 \dots k \}$.
We write $\prod S_i$ to refer to the cartesian-product of sets $S_i$. 
Note that this is the $k$-ary cartesian product, and is not simply repeated applications of the binary cartesian product.
So for example $\prod_{i=1}^3 S_i$ equals $\{ (s_1, s_2, s_3) \mid s_1 \in S_1, s_2 \in S_2, s_3 \in S_3 \}$ and not $\{ ((s_1, s_2), s_3) \mid s_1 \in S_1, s_2 \in S_2, s_3 \in S_3 \}$
We use $S^k$ to refer to $\prod_{i=1}^k S$. 

We use vector notation $\vec{x}$ to refer to a vector of elements $(x_1,\dots, x_k)$, $\vec{x}[i]$ to refer to $x_i$, and $\vec{x}[i \leftarrow x'_i]$ to refer to $\vec{x}$ with $x'_i$ replacing value $x_i$ at position $i$. 
We define $\prod C_i := \{ \prod c_i \mid c_i \in C_i, i \in \{1,\dots,k\} \}$. 
We write $\vec{c}$ for any element of $\prod C_i $ and will often denote $\vec{c}$ by $(c_1, \dots, c_k)$ in place of $\prod c_i$. 

The results below answer the following question:
For what set of queries $Q$ does the learnability of each $C_i$ imply the learnability of $\prod C_i $ and how does the number of queries to learn $\prod C_i $ increase as a function of each $\genCi{i}(c_i)$ for each $q \in Q$?

The proofs in this paper make use of the following simple observations 
\begin{observation}
\label{subobs}
For sets $S_1, S_2, \dots, S_k$ and $T_1, T_2, \dots, T_k$, we have $\prod S_i \subseteq \prod T_i$ if and only if $S_i \subseteq T_i$ for all $i$ or $\prod S_i = \emptyset$.
\end{observation}

\begin{observation}
\label{posobs}
Fix sets $S_1, S_2, \dots, S_k$, points $x_1, x_2, \dots, x_k$ and an index $i$. 
If $x_j \in S_j$ for all $j \ne i$, then $(x_1, x_2, \dots, x_k) \in \prod S_i$ if and only if $x_i \in S_i$.
\end{observation}


\section{Negative Results}
This section introduces some fairly simple lower bounds.

We will start with a lower-bound on learnability from positive examples. 

\begin{proposition}
There exist concepts $C_1$ and $C_2$ that are each learnable from constantly many positive queries, such that $C_1 \times C_2$ is not learnable from any number of positive queries. 
\end{proposition}
\begin{proof}
Let $C_1 := \{ \{a\}, \{a,b\} \}$ and set $C_2 := \{ \mathbb{N}, \mathbb{Z} \backslash \mathbb{N} \}$. 
To learn the set in $C_1$, pose two positive queries to the oracle, and return $\{a,b\}$ if and only if both $a$ and $b$ are given as positive examples. 
To learn $C_2$, pose one positive query to the oracle and return $\mathbb{N}$ if and only if the positive example is in $\mathbb{N}$. 
An adversarial oracle for $C_1 \times C_2$ could give positive examples only in the set $\{a\} \times \mathbb{N}$. 
Each new example is technically distinct from previous examples, but there is no way to distinguish between the sets $\{a\}\times \mathbb{N}$ and $\{a,b\} \times \mathbb{N}$ from these examples. 
\end{proof}

Now we will show lower bounds on learnability from $\eqQ$, $\subQ$, and $\memQ$. 
We will see later that this lower bound is tight when learning from membership queries, but not equivalence or subset queries.


\begin{proposition}
There exists a concept $C$ that is learnable from $\genC$ many queries posed to $Q \subseteq \{ \memQ, \eqQ, \subQ \}$ such that learning $C^k$ requires $(\genC)^k$ many queries.   \todo{Should I explicitly handle infinite and finite cases separately? Should I include bigO notation on the infinite case?}
\end{proposition}
\begin{proof}
Let $C = \{ \{j\} \mid j \in \{0 \dots m\} \}$. 

We can learn $C$ in $m$ membership, subset, or equivalence queries by querying $j \in c^*$, $\{ j \} \subseteq c^*$, or $\{j\} = c^*$, respectively. 

However, a learning algorithm for $C^k$ requires more than $m^k$ queries. 
To see this, note that  $C^k$ contains all singletons in a space of size $(m+1)^k$. 

So for each subset query $\{x\} \subseteq c^*$, if $\{j\} \ne c^*$, the oracle will return $j$ as a counterexample, giving no new information.  
Likewise, for each equivalence query $\{j\} = c^*$, if $\{j\} \ne c^*$, the oracle can return $j$ as a counterexample.
Therefore, any learning algorithm must query $x \in c^*$, $\{ x \} \subseteq c^*$, or $\{x\} = c^*$ for $(m+1)^k - 1$ values of $x$
\end{proof}


\section{Positive Results}

\begin{proposition}
%Assume for each concept class $C_i$ there is a learning algorithm $A_i$ that learns each $c_i \in C_i$ in $$ superset queries (we don't assume $Q_i(c_i)$ is finite for all $c_i$). 

If $Q = \{ \supQ \}$, then there is an algorithm that learns any concept $\targ = c^*_1 \times \dots \times c^*_k  \in \prod C_i $ in $\sum \supCi{i}(c^*_i)$ queries.  
\end{proposition}
\begin{proof}
Algorithm \ref{supalg} learns $\prod C_i $ by simulating the learning of each $A_i$ on its respective class $C_i$. 
The algorithm asks each $A_i$ for subset queries $S_i$, queries the product $\prod S_i$ to the oracle, and then uses the answer to answer at least one query to some $A_i$. 
Since at least one $A_i$ receives an answer for each oracle query, at most $\sum \supCi{i}(c^*_i)$ queries must be made in total. 

 

We will now show that each oracle query results in at least one answer to an $A_i$ query (and that the answer is correct). 
The oracle first checks if the target concept is empty, if not it proceeds as normal.
At each step, the algorithm poses query $\prod S_i$ to the oracle. 
If the oracle returns 'yes' (meaning $\prod S_i \supseteq \targ$), then  $S_i \supseteq c_i^*$ for each $i$ by Observation \ref{subobs}, so the oracle answers 'yes' to each $A_i$. 
If the oracle returns 'no', it will give a counterexample $\vec{x} = (x_1,\dots,x_k) \in \targ \backslash \prod S_i$. 
There must be at least one $x_i \not\in S_i$ (otherwise, $\vec{x}$ would be in $\prod S_i$). 
So the algorithm checks $x_j \in S_j$ for all $x_j$ until an $x_i \not\in S_i$ is found. 
Since $\vec{x} \in \targ$, we know $x_i \in c_i^*$, so $x_i \in c_i^* \backslash S_i$, so the oracle can pass $x_i$ as a counterexample to $A_i$. 

Note that once $A_i$ has output a correct hypothesis $c_i$, $S_i$ will always equal $c_i$, so counterexamples must be taken from some $j \ne i$. 
\end{proof}





\begin{algorithm}[H]
\label{supalg}
\SetAlgoLined
\KwResult{Learn $\prod C_i $ from Superset Queries}
\If{$\emptyset \in C_i$ for some $i$}{
	Query $\emptyset \supseteq \targ$\;
	\If{$\emptyset \supseteq \targ$}{
		\Return{$\emptyset$}
	}
}
\For{$i = 1 \dots k$}{
	Set $S_i$ to initial subset query from $A_i$
}
 \While{Some $A_i$ has not completed}{
  Query $\prod S_i$ to oracle\;
  \eIf{$\prod S_i \supseteq c^*$ }{
   Answer $S_i \supseteq c_i^*$ to each $A_i$\;
   Update each $S_i$ to new query\;
   }{
 	Get counterexample $\vec{x} = (x_1,\dots,x_k)$
   	\For{i = 1 \dots k}{
   		\If{$x_i \not\in S_i$}{
			Pass counterexample $x_i$ to $A_i$\;
			Update $S_i$ to new query\;
			} 
  		}
 	}
	\For{i = 1 \dots k}{
		\If{$A_i$ outputs $c_i$}{
			Set $S_i := c_i$\;
		}
	}
 }
 \Return{$\prod c_i$}\;
 \caption{Algorithm for learning from Subset Queries}
\end{algorithm}






\todo{combine learning algorithm for sub, mem, and eq w/ one positive query}
\begin{proposition}
If $Q = \{\memQ\}$ and a single positive example $\vec{p} \in \targ$ is given, then $\targ$ is learnable in $k \cdot \sum \memCi{i}(c^*_i)$ membership queries. 
\end{proposition}
\begin{proof}
Algorithm \ref{linmemalg} learns by simulating each $A_i$ in sequence, moving on to $A_{i+1}$ once $A_i$ returns a hypothesis $c_i$. 
For any membership query $M_i$ made by $A_i$, $M_i \in c^*_i$ if and only if $\vec{p}[i \leftarrow M_i]\in \targ$ by Observation \ref{posobs}. 
Therefore the algorithm is successfully able to simulate the oracle for each $A_i$, yielding a correct hypothesis $c_i$. 
\end{proof}


\begin{algorithm}[H]
\label{linmemalg}
\SetAlgoLined
\KwResult{Learn $\prod C_i $ from Membership Queries and One Positive Example}
Get positive example $\vec{p}$\;
\For{$i = 1 \dots k$}{
	\While{$A_i$ has not returned a hypothesis $c_i$}{
		Get membership query $M_i$ from $A_i$\;
		Query $\vec{p}[i \leftarrow M_i]$ to oracle\;
		\eIf{Oracle returns 'yes'}{
			Pass answer $M_i \in c^*_i$ to $A_i$\;
		}{
			Pass answer $M_i \not\in c^*_i$ to $A_i$\;
		}
	}
}
\Return{$\prod c_i$} \;

 %\Return{Hi}\;%$\prod c^*_i$\;
 \caption{Algorithm for learning from Membership Queries and One Positive Example}
\end{algorithm}




\begin{algorithm}[H]
\label{linsubalg}
\SetAlgoLined
\KwResult{Learn $\prod C_i $ from Subset Queries, Membership Queries, and One Positive Example}
\For{$i = 1 \dots k$}{
	Set $S_i$ to initial subset query from $A_i$
}
 \While{Some $A_i$ has not completed}{
  Query $\prod S_i \subseteq \targ$ to oracle\;
  \eIf{$\prod S_i \subseteq \targ$}{
  	 Answer $S_i \subseteq c_i^*$ to each $A_i$\;
 	 Update each $S_i$ to new query from $A_i$\;
	}{
 		Get counterexample $\vec{x} = (x_1,\dots,x_k)$\;
		\For{i = 1 \dots k}{
			Query $\vec{p}[i \leftarrow x_i] \in \targ$\;
			\If{$\vec{p}[i \leftarrow x_i] \not\in \targ$ and $x_i \in S_i$}{
				Pass counterexample $x_i$ to $A_i$\;
				Update $S_i$ to new query from $A_i$\;
			}
		}
	
	}
	\For{i = 1 \dots k}{
		\If{$A_i$ outputs $c_i$}{
			Set $S_i := c_i$\;
		}
	}
}
\Return{$\prod c_i$} \;
\caption{Algorithm for learning from Subset Queries, Membership Queries, and One Positive Example}
\end{algorithm}

\begin{proposition}
If $Q = \{\subQ\}$ and a single positive example $\vec{p} \in \targ$ is given, then $\targ$ is learnable in $\sum \subCi{i}(c^*_i)$ subset queries and $k \cdot \sum \subCi{i}(c^*_i)$ membership queries. 
\end{proposition}
\begin{proof}
The learning process is described in Algorithm \ref{linsubalg}.
For each subset query $\prod S_i \subseteq \targ$, the algorithm either returns `yes' or gives a counterexample $\vec{x} = (x_1, \dots, x_k) \in \prod S_i \backslash \targ$. 
If the algorithm returns 'yes', then by Observation \ref{subobs} $S_i \subseteq c^*_i$ for all $i$, so the algorithm can return 'yes' to each $A_i$. 
Otherwise, $\vec{x} \not\in \targ$, so there is an $i$ such that $x_i \not\in c^*_i$. 
By Observation \ref{posobs} the algorithm can query $\vec{p}[j \leftarrow x_j]$ for all $j$ until the $x_i \not\in c^*_i$ is found. 

Once the correct $c_j$ is found for any $j$, $S_j$ will equal $c_j$ for all future queries, so any counterexamples must fail on an $i \ne j$. 

Each subset query results in a correct answer being given to at least one learner $A_i$ and at most $k$ membership queries are made per subset query, yielding the desired bound on queries. 
\end{proof}

\begin{algorithm}[H]
\label{lineqalg}
\SetAlgoLined
\KwResult{Learn $\prod C_i $ from Equivalence Queries, Membership Queries, and One Positive Example}
\For{$i = 1 \dots k$}{
	Set $S_i$ to initial equivalence query from $A_i$
}
  Query $\prod S_i = \targ?$ to oracle\;
  \eIf{$\prod S_i = \targ$ }{
	\Return{$\prod S_i$}	\;
   }{
 	Get counterexample $\vec{x} = (x_1,\dots,x_k)$\;
	\eIf{$\vec{x} \in \targ \backslash \prod S_i$}{
		\For{i = 1 \dots k}{
   			\If{$x_i \not\in S_i$}{
			Pass counterexample $x_i$ to $A_i$\;
			Update $S_i$ to new query from $A_i$\;
			}
  		}
 	}{
		\For{i = 1 \dots k}{
			Query $\vec{p}[i \leftarrow x_i] \in \targ$\;
			\If{$\vec{p}[i \leftarrow x_i] \not\in \targ$ and $x_i \in S_i$}{
				Pass counterexample $x_i$ to $A_i$\;
				Update $S_i$ to new query from $A_i$\;
			}
		}
	
	}
}
\caption{Algorithm for learning from Equivalence Queries, Membership Queries, and One Positive Example}
\end{algorithm}


Finally, we study the case when $Q = \{\eqQ\}$, as described in Algorithm \ref{lineqalg}.
This algorithm works as a synthesis of the learning algorithms for Supersets and Subsets. 
When a negative example is given, the algorithm runs as in Algorithm \ref{linsubalg} for handling subset queries. 
When a positive example is given, the algorithm runs as in Algorithm \ref{supalg} for handling superset queries. 

\section{Disjoint Union}
This section discusses learning disjoint unions of concept classes. 
This is generally much easier than learning cross-products of classes, since counterexamples belong to a single dimension in the disjoint union. 
This problem uses the same notation as the cross-product case, but we denote the disjoint union of two sets as $A \cupdot B$ and the disjoint union of many sets as $\bigcupdot A_i$.  
We define the concept class of disjoint unions as $\disClass := \{ \bigcupdot c_i \mid c_i \in C_i  \}$. 

The algorithm for learning from membership queries is very easy and won't be stated here. 
Algorithm \ref{disjalg} shows the learning procedure for when $Q \in \{ \{\subQ\}, \{\supQ\}, \{\eqQ\}\}$.
The correctness of this algorithm follows from the following simple facts.
Assume we have sets $S_1,\dots,S_k$ and $T_1,\dots,T_k$.
Then $\bigcupdot S_i \subseteq \bigcupdot T_i$ if and only $S_i \subseteq T_i$ for all $i$.
Likewise $\bigcupdot S_i = \bigcupdot T_i$ if and only if $S_i = T_i$ for all $i$.



\begin{algorithm}[H]
\label{disjalg}
\SetAlgoLined
\KwResult{Learning Disjoint Unions}
\For{$i = 1 \dots k$}{
	Set $S_i$ to initial query from $A_i$
}
\While{Some $A_i$ has not terminated}{
  Query $\bigcupdot S_i$ to oracle\;
  \eIf{Oracle returns `yes' }{
	Pass 'yes' to each $A_i$\;
   	Get updated $S_i$ from each $A_i$\; 
   }{
 	Get counterexample $x_i \in X_i$ for some $i$\;
	Pass $x_i$ as counterexample to $A_i$\;
	Get updated $S_i$ from each $A_i$\; 
	}
}
\Return{$\bigcupdot S_i$}\;
\caption{Learning Disjoint Unions}
\end{algorithm}





\section{Learning with Only Membership Queries}

We have seen that learning with membership queries can be made significantly easier if a single positive example is given. 
In this section we describe a learning algorithm using membership queries when no positive example is given. 
This algorithm makes $O(max_i \{ \memCi{i}(c_i) \}^k)$ queries, matching the lower bound given in a previous section. 

For this algorithm to work, we need to assume that $\emptyset \not\in C_i$ for all $i$.
If not, there is no way to distinguish between an empty and non-empty concept. 
For example consider the classes $C_1 = \{ \{1\}, \emptyset \}$ and $C_2 = \{ \{j \} \mid j \in \mathbb{N} \}$. 
It is easy to know when we have learned the correct class in $C_1$ or in $C_2$ using membership queries. 
However, for any finite number of membership queries, there is no way to distinguish between the sets $\emptyset$ and $\{(1,j)\}$ for some $j$ that has yet to be queried.


The main idea behind this algorithm is that learning from membership queries is easy once a single positive example is found. 
So the algorithm runs until a positive example is found from each concept or until all concepts are learned. 
If a positive example is found, the learner can then run Algorithm \ref{lineqalg} for learning from membership queries and a single positive example. 


\begin{proposition}
Algorithm \ref{memonlyalg} will terminate after making $O(max_i \{ \memCi{i}(c_i) \}^k)$ queries.
\end{proposition}
\begin{proof}
The algorithm works by constructing sets $S_i$ of elements and querying all possible elements of $\prod S_i$. 
We will get our bound of $O(max_i \{ \memCi{i}(c_i) \}^k)$ by showing the algorithm will find a positive example once $| S_i | > max_i \{ \memCi{i}(c_i) \}$ for all $i$. 
Since the algorithm queries all possible elements of $\prod S_i$, it is sufficient to prove that $S_i$ will contain an element of $c_i$ once $|S_i| > \memCi{i}(c_i)$. 

Assume that each learner eventually terminates. 
%This means that for any $c_i$ in any $C_i$, either $A_i$ eventually queries a point $x \in c_i$ or the learner outputs $c_i$ after finitely many negative membership queries. 
Let $\vec{q}^i = q_1^i, q_2^i, \dots$ be the membership queries $A_i$ makes assuming it only receives negative answers from an oracle. 
If $\vec{q}^i$ is finite, then there is some set $N_i \in C_i$ that $A_i$ outputs after querying all points in $\vec{q}^i$ (and receiving negative answers). 
If $N_i$ is non-empty let $n_i$ be some element in $N_i$. 
Note that although sampling elements from a set might be expensive in general, this is only done for $N_i$ and can therefore be hard-coded into the learning algorithm. 
If $c_i = N_i$, then by our assumption that $\targ \ne \emptyset$, $N_i$ contains some $n_i$. 
So $S_i$ contains an element of $c_i$ at the start of the algorithm. 
If $c_i \ne N_i$, by our assumption that $A_i$ eventually terminates, $A_i$ must eventually query some $q_j^i \in c_i$. 
So after $j$ steps, $S_i$ contains some element of $c_i$.
Since $j < \memCi{i}(c_i)$, we have that $S_i$ contains a positive example once $| S_i | > \memCi{i}(c_i)$, completing the proof.  
\end{proof}






\begin{algorithm}[H]
\label{memonlyalg}
\SetAlgoLined
\KwResult{Learning with Membership Queries Only}
\For{$i = 1 \dots k$}{
	\eIf{$N_i$ and $n_i$ exist}{
		Set $S_i := \{n_i\}$\;
	}{
		Set $S_i := \{ \}$\; 
	}
}
Set $j = 0$\;
\While{True}{
	\For{i = \{1, \dots, k\}}{
		\If{ $|\vec{q}^i| \ge j$ }{
			$S_i := S_i \cup \{q_j^i\}$\;
		}
	}
  \For{$\vec{x} \in \prod S_i$}{
  	Query $\vec{x} \in \targ$\;
	\If{$\vec{x} \in \targ$}{
		Run Algorithm \ref{linmemalg} using $\vec{x}$ as a positive example\;	
	}
  }
  $j := j+1$\;
}
\caption{Algorithm for Learning from Membership Queries Only}
\end{algorithm}



\section{Learning Cartesian Products with Equivalence or Subset Queries is Hard}


The previous section showed that learning cross products of membership queries requires at most $O(max_i \{ \memCi{i}(c_i) \}^k)$ membership queries. 
A natural next question is whether this can be done for equivalence and subset queries. 
In this section, we answer that question in the negative. 
We will construct a class $\eqhard$ that can be learned from $n$ equivalence or membership queries but which requires at least $k^n$ queries to learn $\eqhard^k$.  

We define $\eqhard$ to be the set $\{ \ntreef(s) \mid s \in \mathbb{N}^* \}$, where $\ntreef(s)$ is defined as follows:

\[\ntreef(\lambda) = \{\lambda\} \times \mathbb{N}\]
\[\ntreef(s) = (\{s\} \times \mathbb{N}) \cup \ntree(s)\]
\[\ntree(sa) = (\{s\} \times (\mathbb{N} \backslash \{a\})) \cup \ntree(s)\]\\

For example, $\ntreef(12) = (\{12\} \times \mathbb{N}) \cup (\{1\}\times(\mathbb{N} \backslash \{2\}))\cup(\{\lambda\} \times (\mathbb{N} \backslash \{1\}))$.


An important part of this construction is that for any two strings $s,s' \in \mathbb{N}$, we have that $\ntreef(s) \subseteq \ntreef(s')$ if and only if $s = s'$. 
This implies that a subset query will return true if and only if the true concept has been found. 
Moreover, an adversarial oracle can always give a negative example for an equivalence query, meaning that oracle can give the same counterexample if a subset query were posed. 
So we will show that $\eqhard$ is learnable from equivalence queries, implying that it is learnable from subset queries. 

We we prove a lower-bound on learning $\eqhard^k$ from subset queries from an adversarial oracle. 
An adversarial equivalence query oracle can give the exact same answers and counterexamples, implying that  $\eqhard^k$ is hard to learn from equivalence queries. 

\begin{proposition}
There exist algorithms for learning from equivalence queries or subset queries such that any concept $\ntreef(s) \in \eqhard$ can be learned from $|s|$ queries. 
\end{proposition}
\begin{proof}
(proof sketch) Algorithm \ref{ntree} shows the learning algorithm for equivalence queries. 
As mentioned above, this algorithm is essentially the same for learning from subset queries. 
When learning $\ntreef(s)$ for any $s \in \mathbb{N}^*$, the algorithm will construct $s$ by learning at least one new element of $s$ per query. 
Each new query to the oracle is constructed from a string that is a substring of $s$
If a positive counterexample is given, this can only yield a longer substring of $s$.
\end{proof}


\begin{algorithm}[H]
\label{ntree}
\SetAlgoLined
\KwResult{}
Set $s = \lambda$\;
\While{True}{
	Query $\ntreef(s)$ to Oracle
	\If{Oracle returns `yes'}{
		\Return $\ntreef(s)$
	}
	\If{Oracle returns $(s',m) \in c^* \backslash \ntreef(s)$ }
	{
		Set $s = s'$\;
	}
	\If{Oracle returns $(s,m) \in \ntreef(s) \backslash c^*$}{
		Set $s = sm$\;
	}
} 
\caption{Learning $\eqhard$ from equivalence queries.}
\end{algorithm}


\subsection{Showing $\eqhard^k$ is Hard to Learn}

It is easy to learn $\eqhard$, since each new counterexample gives one more element in the target string $s$. 
When learning a concept, $\prod \ntreef(s_i)$, it is not clear which dimension a given counterexample applies to. 
Specifically A given counterexample $\vec{x}$ could have the property that $\vec{x}[i] \in \ntreef(s_i)$ for all $i \ne j$, but the learner cannot infer the value of this $j$. 
It will then proceed considering all possible values of $j$, requiring exponentially more queries for longer $s_i$. \todo{is this clear?} 
This subsection will formalize this notion to prove an exponential lower bound on learning $\eqhard^k$. 
First, we need a couple definitions. 


A concept $\prod \ntreef(s_i)$ is \emph{justifiable} if one of the following holds:
\begin{itemize}
\item For all $i$, $s_i = \lambda$
\item There is an $i$ and an $a \in \mathbb{N}$ and $w \in \mathbb{N}^*$ such that $s_i = wa$, and $\ntreef(s_1) \times \dots \times \ntreef(w) \times \dots \times \ntreef(s_k)$ \todo{need flat cross product} was justifiably queried to the oracle and received a counterexample $\vec{x}$ such that $\vec{x}[i] = (w, a)$. 
\end{itemize}

A concept is \emph{justifiably queried} if it was queried to the oracle when it was justifiable. 
\newline


The adversarial oracle works as follows:
\begin{itemize}
\item It will always answer the same query with the same counterexample.  
\item Given any query $\prod \ntreef(s_i) \subseteq c^*$, the oracle will return a counterexample $\vec{x}$ such that for all $i$, $\vec{x}[i] = (s_i, a_i)$, and $a_i$ has not been in any query or counterexample yet seen.
\end{itemize}

Consider the following example when $k = 2$. 
First, the learner queries $(\ntreef(\lambda), \ntreef(\lambda))$ to the oracle and receives a counter-example $((\lambda, 1), (\lambda, 2))$. 
The justifiable concepts are now $(\ntreef(1), \ntreef(\lambda))$ and $(\ntreef(\lambda), \ntreef(2))$. 
The learner queries $(\ntreef(1), \ntreef(\lambda))$ and receives counterexample  $((1, 3), (\lambda, 4))$. 
 The learner queries $(\ntreef(\lambda), \ntreef(2))$ and receives counterexample $((\lambda, 5), (2, 6))$.
The justifiable concepts are now  $(\ntreef(1), \ntreef(4))$, $(\ntreef(1 \cdot 3), \ntreef(\lambda))$, $(\ntreef(5), \ntreef(2))$ and $(\ntreef(\lambda), \ntreef(2 \cdot 6))$.  
At this point, the only possible solutions constructible from strings of length $1$ are $(\ntreef(1), \ntreef(4))$ and $(\ntreef(5), \ntreef(2))$. 





For any strings $s,s' \in \mathbb{N}^*$, we write $s \le s'$ if $s$ is a substring of $s'$, and we write $s < s'$ if $s \le s'$ and $s \ne s'$.
We say that the \emph{sum of string lengths} of a concept $\prod \ntreef(s_i)$ is of size $r$ if $\sum |s_i| = r$

The following simple proposition can be proven by induction on sum of string lengths.

\begin{proposition}
\label{subjust}
Let $\prod \ntreef(s_i)$ be a justifiable concept. 
Then for all $w_i \le s_i$, $\prod \ntreef(w_i)$ has been queried to the oracle.
\end{proposition}

\begin{proposition}
\label{numjustconc}
If all justified concepts $\prod \ntreef(s_i)$ with sum of string lengths equal to $r$ have been queried, then there are $k^{r+1}$ justified queries whose sum of string lengths equals $r+1$
\end{proposition}
\begin{proof}
This proof follows by induction on $r$. 
When $r=0$, the concept $\prod \ntreef(\lambda)$ is justifiable.%has been queried and a counterexample $\vec{x}$ has been given.
%Then for all $i$, the concept which is $\ntreef(\lambda)$ at all $j \ne i$ and $\ntreef(\vec{x}[i])$ is justifiable (there are $k$ such $i$.
For induction, assume that there are $k^r$ justifiable queries with sum of string lengths equal to $r$. 
By construction, the oracle will always chose counterexamples with as-yet unseen values in $\mathbb{N}$. 
So querying each concept $\prod \ntreef(s_i)$ will yield a counterexample $\vec{x}$ where for all $i$, $\vec{x}[i] = (s_i, a_i)$ for new $a_i$.
Then for all $i$, this query creates the justifiable concept $\prod \ntreef(s'_i)$, where $s'_j = s_j$ for all $j \ne i$ and $s'_i = \ntreef(s_i \cdot a_i)$.
Thus there are $k^{r+1}$ justifiable concepts with sum of string lengths equal to $r+1$.
\end{proof}

\begin{theorem}
Any algorithm learning $\eqhard^k$ from subset queries requires at least $k^r$ subset (or equivalence) queries to learn a concept $\prod \ntreef(s_i)$, whose sum of string lengths is $r$.
\end{theorem}
\begin{proof}
Assume for contradiction that an algorithm can learn with less than $k^r$ queries and let this algorithm converge on some concept $c = \prod \ntreef(s_i)$ after less than $k^r$ queries. \todo{explain how to assume the oracle will never return "true" on a query.}
%We will construct another concept $c' = \prod \ntreef(s'_i)$ that is consistent with all given oracle answers, but whose sum of string lengths is less than $r$.
Since less than $k^r$ queries were made to learn $c$, by Proposition \ref{numjustconc}, there must be some justifiable concept $c' = \prod \ntreef(s'_i)$ with sum of string lengths less than or equal to $r$ that has not yet been queried.
By proposition \ref{subjust} we can assume without loss of generality that for all $w_i \le s_i'$, $\prod \ntreef(w_i)$ has been queried to the oracle.
We will show that $c'$ is consistent with all given oracle answers, contradicting the claim that $c$ is the correct concept. 
Let $c_v = \prod \ntreef{v_i}$ be any concept queried to the oracle, and let $\vec{x}$ be the given counterexample.
If for all $i$, $v_i \le s'_i$, then by construction, there is an $i$ with $\vec{x}[i] = (v_i, a_i)$ such that $v_i \cdot a_i \le s'_i$, so $\vec{x}$ is a valid counterexample.
Otherwise, there is an $i$ such that $v_i \not\le s'_i$. 
So $\{v_i\} \times \mathbb{N}  \cap \ntreef(s'_i) = \emptyset$, so $\vec{x}$ is a valid counterexample. 
Therefore, all counterexamples are consistent with $c'$ being correct concept, contradicting the claim that the learner has learned $c$. 
\end{proof}

\end{document}  