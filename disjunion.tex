This section discusses learning disjoint unions of concept classes. 
This is generally much easier than learning cross-products of classes, since counterexamples belong to a single dimension in the disjoint union. 
This problem uses the same notation as the cross-product case, but we denote the disjoint union of two sets as $A \cupdot B$ and the disjoint union of many sets as $\bigcupdot A_i$.  
We define the concept class of disjoint unions as $\disClass := \{ \bigcupdot c_i \mid c_i \in C_i  \}$. 

The algorithm for learning from membership queries is very easy and won't be stated here. 
Algorithm \ref{disjalg} shows the learning procedure for when $Q \in \{ \{\subQ\}, \{\supQ\}, \{\eqQ\}\}$.
The correctness of this algorithm follows from the following simple facts.
Assume we have sets $S_1,\dots,S_k$ and $T_1,\dots,T_k$.
Then $\bigcupdot S_i \subseteq \bigcupdot T_i$ if and only $S_i \subseteq T_i$ for all $i$.
Likewise $\bigcupdot S_i = \bigcupdot T_i$ if and only if $S_i = T_i$ for all $i$.



\begin{algorithm}[H]
\label{disjalg}
\SetAlgoLined
\KwResult{Learning Disjoint Unions}
\For{$i = 1 \dots k$}{
	Set $S_i$ to initial query from $A_i$
}
\While{Some $A_i$ has not terminated}{
  Query $\bigcupdot S_i$ to oracle\;
  \eIf{Oracle returns `yes' }{
	Pass 'yes' to each $A_i$\;
   	Get updated $S_i$ from each $A_i$\; 
   }{
 	Get counterexample $x_i \in X_i$ for some $i$\;
	Pass $x_i$ as counterexample to $A_i$\;
	Get updated $S_i$ from each $A_i$\; 
	}
}
\Return{$\bigcupdot S_i$}\;
\caption{Learning Disjoint Unions}
\end{algorithm}



